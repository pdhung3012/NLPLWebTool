\specialchapt{ABSTRACT}

Computer Programming should be considered as art . Literate Programming (LP) (\cite{Knuth:1984:LP:473.479})  is a programming paradigm proposed by Donald Knuth
to realize this idea. LP helps developers to make the source code to be more like a literature, by the ability of 
integrate the explanation in natural language (NL) along with original source code. However, since its appearance in 1984, 
LP has not been used as a popular programming paradigm. One of the challenges is that developers need to write 
the corresponding source code manually for any NL parts they defined to make the program compilable. 
In this project, we proposed Natural Language to Programming
Language (NLPL) system, to reduce the effort of writing code. 
Our NLPL system allows developers to write explaination as code comment in NL and automatically
generating the respected source code following the requirements of the NL parts. The code generation is done by the NLVisitor module we developed
for providing translation rules at Natural Language syntax level and alleviating indirect references problems between languages.
The experiment on the set of 52 comments from the code suggestion tool AnyCode (\cite{Gvero:2015:SJE:2814270.2814295}) achieves the result at 75\% in top-1 accuracy, 
which outperforms this prior work (62\%).  