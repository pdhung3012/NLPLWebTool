% Chapter 3 from the thesis template file
%   that contains an example table and figure.
\chapter{METHODS AND PROCEDURES}
In the previous chapter, we discuss main problems of NL to PL translation and elements of NL Parser, which are different types of tags for building an NL parse tree representation for NL sentence. These knowledge help us for gaining background to design our syntax-based approach for NL to PL translation. while NLP researches relies on analyzing NL parsed tree, we also considered NL parsed tree for NL description as the core element for translation. To handle the syntax of NL parsed tree, we relied on how programming language handled an Abstract Syntax Tree (AST) which is a representation of a source code file by building ASTVisitor. Based on ideas of ASTVisitor in PL, we designed the NLPLVisitor for NL to PL translation.  
\section{Designing a NLPL Visitor}
\subsection{Visitor Pattern}
In programming language design, collections are most important data types. A collection mays have instances from multiple types. Since we usually use loop for handling information from collection, it is very common that we need to write an operation or method to handle each element of collection without knowing the type of each elements. There is one way to write an operation, which used "instance of" condition to check the type of each elements in a collection, however this way is not reflecting object oriented design. 

A pattern that allows handle element of different types inside a collection in object oriented way calls Visitor pattern. In a visitor pattern, developers have the ability for representing operations on each types of elements of a collection. Besides, it allows you to write new operation without interfering the previous operations you define for types of element. In other words, visitor pattern overcomes the conflict between object structure s and algorithm  which the the object structure operates on, means you can update the algorithm handling for each elements without changing the structure of element. Along with collections, visitor pattern is also available for other data type structures such as tree structure.

One of the most useful data structure for handling source code is Abstract-Syntax- Tree (AST). In AST tree, each node in tree belong to a specific node type depend on the source code. An example of AST is shown on figure \cite{}. In this example, "println" is a method invocation node which contains elements as its child node like a String literal argument. Obviously, to handle content of AST, we need to have an visitor pattern for taking operation on each type of nodes. ASTVisitor, which is developed by eclipse JDT (\cite{})is an important implementation of visitor pattern for operating the AST tree.

A design diagram of ASTVisitor is shown as follows. ASTVisitor is the name of the interface in package of eclipse JDT. It has several operation defined for each type of AST node, the most important operation is the visit() method declaration. There are set of method visit() in ASTVisitor, which each method take a specific type of ASTNode as type of its argument and each method will execute some algorithms based on type of ASTNode. To handle the operation on ASTNode on our own, we need to define a class that implements ASTVisitor and define content inside body of each visit operation. Our own visitor can be passed as an argument of ASTParser, which is used to produce the ASTNode for the whole source code and storing information of our visitor. A full class diagram of an ASTVisitor is shown on figure \cite{}.
 
\subsection{NLPLVisitor}
Inspiring from advantages of ASTVisitor in programming language, we use a visitor pattern for design a module for dealing with the complexity of Natural Language parsed tree in NL to PL translation. NL parsed tree, to be remind, is a data structure for representing NL description in the form of a tree in which each node belongs to a specific tags in a tag sets of Penn Tree bank we discussed in the previous chapter. Similar to handling AST tree in PL, we design and implement an abstract class NLPL Visitor which we defined the basic visit method declaration for all types of NL tags. Then, we defined implementation of each visit operation relates to each type by writing a class that implements the NLPL visitor. Since our work is for translating between two languages, we called this implemented class Translation Visitor. 

Considering the translation can be done by a syntax based approach, we implement the content of each visit function as follows. First, we need to understand the intuition of each NL tags in the translation to PL. Next, we provide rules for each NL tags for translating to get  elements in PL. In this step, we considered the children of each NL node and its terminal (textual value) as the most important information to decide which rules will be used for the translation of the node. The intuition of each NL tags is provided based on our observations on the NL description corpus as documentation of Java Development Kit (JDK), and the summarization of them are shown in figure \cite{}. The implemented rules for each NL tags will be discussed in the next section.

\section{Solving Problems between Natural Language and Programming Language}
\subsection{Context: Extracting Information from Variable and Literal}
Context of NL description is information that can be mentioned in it. We consider surrounding code of NL description in a source code file and al Java APIs that can be accessible by a Java project that contains the description are two resources for representing the context of source code. In NLPL, given input of java project, NLPL has a module for extracting all accessible APIs in this project. They are APIs from imported library and APIs defined in each source files. The algorithm for extracting all accessible APIs is shown in figure \cite{}. \\
In NL side,an NL description usually mentions about information of variable and literal value along with class name. We consider ambiguous types of class name is one type of indirect reference and we handle that by a reasoning module in the next section. For variables, we get information of all accessible variables in a set of variable object extract by algorithm \cite{} and check in NL description if it mentioned about a variable or not. For literal, we check if NL description contains a literal value by regular expression. We define regular expressions for string value and numeric value to do the checking. \\
One problems of NL parser is that it relied on probabilistic parsing, so it can be error-prone. By looking at examples of NL descriptions and see the results of NL parser manually, we identify that keeping variable and literal in original form caused parsing with not useful information. For example, in the code showed in figure \cite{}, we see that a string value "text.txt" broke into 3 parts in NL parser, while we should consider it as a single element for processing. Similarly, with example 2 on figure \cite{}, the variable "a" is parsed to be a node of verb, which is not correct. To overcome this problem which cause by non-useful tags and incorrect tags, we represented variables and literal as an unique alias by the last step of preprocessing step in algorithm \cite{}. An example of unique alias is shown in the example of algorithm.  

\subsection{NL Tree Transformation}
The appearance of variable and literal inside NL description leads to our observation is that, in NL descriptions, some part can be considered as textual node while other parts are actually expressions like variable and literal. The expression parts can be used directly when composing to the final results for translation. This fact motivates us to transform the original parsed tree to a transformed tree with 2 types of nodes: Textual node and Expression node. This transformation allows us to embedding information specifically depends on node. An example of NL Tree transformation is shown on figure \cite{}.  In this example, we are able to embed information about variable a such as its type, its scope  as local variable inside an Expression node. The class diagram of Textual Node and Expression is shown on figure \cite{}. Both Expression node and Textual node will have information about NL tags, they keep all informations from original NL parsed tree. Since the final output for NLPL is source code which can be expressions or statement, the transformed tree served as a representation of a bridge language which connect information between source language and target language. The idea of bridge language has been applied in Natural Language translations such as for English, French or Russian \cite{}.

\subsection{Phrase: Connecting information between each Phrase in translation}
We handle complexity in phrase level by the idea that we need to find a way to combine translation results of each phrases in NL description. We consider conjunction tags CC, such as "and" or "or" are two main elements that separate between each phrase in translation.  In NL description, we consider that both "and" and "or" are usually used between boolean literal for expressing logical operator. In addition, "and" is usually used for combining translated results of each single phrase. We handle the "and" and the "or" conjunctions by algorithm shown in figure \cite{}. \\
An example of different usage of "and" conjunction is shown by 3 examples in figure \cite{}. In the first example, we have "and" appears between two  boolean expressions that are used for comparison. In this case, due to the translated result is boolean for each single phrase, then the final translated result of "and" will be "\&\&" operator. In the second example, we see that the translated result for the first single clause is a constructor and the type of the constructor is the type of receiver of method API in the second clause. In this case, the combineExpression() function will mix translated results of these two phrases into a single expression. The thrid example shows a different condition compared to example 2. In this case, the constructor in the first clause doesn't have the same type with neither receiver and arguments. In this case the combineExpression() function will return 2 expressions consecutively as the final translated result.  \\
In the first 3 sections, we propose algorithms for take advantages of context in NL to PL translation and reasoning on on complex phrases, which are two per six main problems of naturalistic programming (\cite{}). In next sections, we will introduce our solution for solving four remaining problems which relate to Indirect reference: deixis, expressiveness, anaphoric relation and ambiguity.
\subsection{Rules for translating Natural Language elements}
In the NLPLVisitor, we provide rules for handling each NL tags. The main idea is that we identify types of tags, what children tags it produce then provide rules and its terminal value. An overview of rules and function relate to each tags is shown in table \ref{}.

\makeatletter
\def\BState{\State\hskip-\ALG@thistlm}
\makeatother
\begin{algorithm}
	\caption{Algorithm for Translation at Clause Level}
	\label{alg:TranslateClause}
	\begin{algorithmic}
		\Require \\
		 \textit{node} - NLP node in one of types: S, 	SBAR, SBARQ, SINV, SQ	\\
		 \textit{scope} - information about APIs and variables surround an NL description \\
		  \textit{k} - size of return translated options
		\Ensure $listResults$ - list of all possible code options as translated results  
		\Function {translateClause}{$node$}
		\State {$\textit{listResults} \gets \textit{\{\}};$}
		\If {  $node$ is a sentence }
		\If {  $node$ is a single clause }
		\State {$\textit{verbNode} \gets \text{the verb inside } \textit{node};$} 
		\State {$\textit{listTerms} \gets \text{all terms inside } \textit{node};$} 
		\State {$\textit{listExprs} \gets \text{all expressions inside } \textit{node};$} 
		\State {$\textit{listResults} \gets \textit{getAllRelatedAPIs(verbNode,listTerms,listExprs,k)};$}
		\Else 
		\State {$\text{// a sentence contains multiple clauses which separated by conjunctions "and" or "or"}$}
		\State {$\textit{listSubClauses} \gets \text{split sentence by multiple clauses } \textit{node};$} 
		\State {$resultI$ $\gets$ {${\{\}}$}}
		\ForEach {$clause \in listSubClauses $}
			\State {$\textit{listAPIs} \gets \textit{getAllRelatedAPIs(clause)};$}
		\State {$\textit{resultI} \gets \textit{resultI}+\textit{listTransI.get(k)};$}
		\EndFor
		\State {$\textit{listResults.add(resultI)};$}
		\EndIf
		\Else 
		\State {$\textit{listSubSentences} \gets \text{split clause by multiple sentences };$} 
		\For{$i \gets 1$ to $k$}                    
			\State {$resultI$ $\gets$ {${\{\}}$}}
			\ForEach {$sentence \in listSubSentences $}
			\State {$\textit{listTransI} \gets \textit{translateClause(sentence)};$}
			\State {$\textit{resultI} \gets \textit{resultI}+\textit{listTransI.get(k)};$}
			\EndFor
			\State {$\textit{listResults.add(resultI)};$}
		\EndFor
		\EndIf
		\State {$\text{return } \textit{listResults};$}
		\EndFunction
	\end{algorithmic}
\end{algorithm}

\begin{algorithm}
	\caption{Algorithm for get all related method APIs from verbs and surrounding context}
	\label{alg:GetAllMethodAPIs}
	\begin{algorithmic}
		\Require \\
		\textit{verbNode} - NLP node in type VP	\\
		\textit{listTerms} - list of nlp node as terms \\
		\textit{listExprs} - list of nlp node as expressions \\
		\textit{k} - size of return translated options
		\Ensure $listResults$ - list of all possible code options as translated results  
		\Function {getAllRelatedMethodAPIs}{$verbNode,listTerms,listExprs,k$}
		\State{$\textit{listIdentifiers} \gets \textit{getListIdentiers(verbNode,listTerms,k)}$}
		\State{$\textit{listMethodAPIs} \gets \textit{getBestMethodSignature(verbNode,listTerms,k) }$}
		\State {$\text{return } \textit{listMethodAPIs};$}
		\EndFunction
	\end{algorithmic}
\end{algorithm}


\begin{algorithm}
	\caption{Algorithm to Add Terms to Ancestor}
	\label{alg:addTermsToAncestor}
	\begin{algorithmic}
		\Require \\
		\textit{node} - NLP node in phrase levels and word levels
	\Function {addTermsToAncestor}{$node$}
	\State {$\textit{listTypesOfAncestor} \gets \text{S, SBAR, SBARQ, SINV, SQ,  ROOT }$} 
	\State {$\textit{ancestor} \gets \text{node}$}
	\While {  $\text{type of } \textit{ancestor} \text{ does not belong to } \textit{listTypesOfAncestor} $}
	\State{$\textit{ancestor} \gets \textit{ancestor.parent};$}
	\EndWhile	
	\State{$\textit{ancestor.listTerms.add(node)};$}
	\EndFunction	
	\end{algorithmic}
\end{algorithm}

\begin{algorithm}
	\caption{Algorithm to Add Verb Phrase to Ancestor}
	\label{alg:addVerbPhraseToAncestor}
	\begin{algorithmic}
		\Require \\
		\textit{node} - NLP node of type Verb Phrase
		\Function {addVerbPhraseToAncestor}{$node$}
		\State {$\textit{listTypesOfAncestor} \gets \text{S, SBAR, SBARQ, SINV, SQ,  ROOT }$} 
		\State {$\textit{ancestor} \gets \textit{node}$}
		\While {  $\text{type of } \textit{ancestor} \text{ does not belong to } \textit{listTypesOfAncestor} $}
		\State{$\textit{ancestor} \gets \textit{ancestor.parent};$}
		\EndWhile	
		\State{$\textit{ancestor.verbPhrase} \gets \textit{node};$}
		\EndFunction	
	\end{algorithmic}
\end{algorithm}

\begin{algorithm}
	\caption{Algorithm to Add Verb Content to Ancestor}
	\label{alg:addVerbContentToAncestor}
	\begin{algorithmic}
		\Require \\
		\textit{node} - NLP node of type VB, VBZ, VBD
		\Function {addVerbContentToAncestor}{$node$}
		\State {$\textit{ancestor} \gets \textit{node}$}
		\While {$\text{type of } \textit{ancestor} \text{ is not VP }$}
		\State{$\textit{ancestor} \gets \textit{ancestor.parent};$}
		\EndWhile
		\If {  $\textit{ancestor}$ is null }
			\State {$\textit{listTypesOfAncestor} \gets \text{S, SBAR, SBARQ, SINV, SQ,  ROOT }$} 
			\State {$\textit{ancestor} \gets \textit{node}$}
			\While {$\text{type of } \textit{ancestor} \text{ does not belong to } \textit{listTypesOfAncestor} $}
			\State{$\textit{ancestor} \gets \textit{ancestor.parent};$}
			\EndWhile
		\EndIf			
		\State{$\textit{ancestor.verbPhrase} \gets \textit{node};$}
		\EndFunction	
	\end{algorithmic}
\end{algorithm}

\begin{algorithm}
	\caption{Algorithm to Add Conjunctions to Ancestor}
	\label{alg:addConjunctionsToAncestor}
	\begin{algorithmic}
  		\Require \\
		\textit{node} - NLP node in conjunction types	
		\Function {addConjunctionsToAncestor}{$node$}
		\State {$\textit{listTypesOfAncestor} \gets \text{S, SBAR, SBARQ, SINV, SQ,  ROOT }$} 
		\State {$\textit{ancestor} \gets \textit{node}$}
		\While {  $\text{type of } \textit{ancestor} \text{ does not belong to } \textit{listTypesOfAncestor} $}
		\State{$\textit{ancestor} \gets \textit{ancestor.parent};$}
		\EndWhile	
		\State{$\textit{ancestor.listConjunctions.add(node)};$}
		\EndFunction	
	\end{algorithmic}
\end{algorithm}

\begin{algorithm}
	\caption{Algorithm to Handle PP before visit}
	\label{alg:checkPP}
	\begin{algorithmic}
		\Require \\
		\textit{node} - NLP node in PP types	
		\Function {checkPP}{$node$}
		\If {  $\text{node has 2 words in child nodes and they are "from console"}$ }
			\State{$\textit{parent} \gets \textit{node.parent}$}
			\State{$\textit{consoleAPI} \gets \text{create an Expression node for } \textit{System.in}$}
			\State{$\text{replace } \textit{node} \text{ by }\textit{consoleAPI} \text{ as child of } \textit{parent}$}
		\EndIf	
		\EndFunction	
	\end{algorithmic}
\end{algorithm}

\begin{algorithm}
	\caption{Algorithm to Check Class Name And Replace Node}
	\label{alg:checkClassName}
	\begin{algorithmic}
		\Require \\
		\textit{node} - NLP node in NP type	\\
		\textit{scope} - information inside Java project \\
		\textit{mapClassNames} - map of class name in which keys are class name and values or list of APIs that have the same class name
		\Function {checkCLassAPI}{$node,scope,mapClassNames$}
		\If {  $\textit{isNodeMentionAboutCLassName(node,mapClassNames)}$ }
		\State{$\textit{strClassName} \gets \text{textual content of } \textit{node}$}
		\State{$\textit{listClassAPIs} \gets \textit{getAllAPIsHaveSameClassName(strClassName,mapClassNames)}$}
		\State{$\textit{strPredictedClassAPI} \gets \textit{getMostFrequentAPIInScope(listClassAPIs,scope)}$}
		\State{$\textit{nodeClassAPI} \gets \text{ create new node of with Class Name annotation from } \textit{strPredictedClassAPI}$}
		\State{$\textit{parent} \gets \textit{node.parent}$}
		\State{$\text{replace } \textit{node} \text{ by }\textit{nodeClassAPI} \text{ as child of } \textit{parent}$}
		\EndIf	
		\EndFunction	
		\end{algorithmic}
	\end{algorithm}


\subsection{Solving IR on Noun Phrase as Class Name}
\subsubsection{Identifying  Class Name as Noun Phrase}
\subsubsection{Resolving Types for  Class Name}
\subsection{Solving IR on Verb Phrase as Synonyms}
\subsection{Solving IR on Verb Phrase as Statement Trigger}
\subsection{Solving IR on Verb Phrase as Method Invocation}
\subsubsection{Constructing Database of Method Invocation}
\subsubsection{Allowing IR of Method Invocation at API levels}
\subsubsection{Algorithm for choosing best Method APIs
}
\subsection{Solving IR on Preposition as Method APIs}



